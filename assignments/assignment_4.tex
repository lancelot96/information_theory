\documentclass{ctexart}

\usepackage{float}
\usepackage{amsmath}
\usepackage{geometry}

\geometry{top=2.5cm, bottom=2.5cm, left=2.5cm, right=2.5cm}

\title{信息论作业 4}
\author{史泽宇}
\date{\today}

\begin{document}

\maketitle

\paragraph{题目 1}

\begin{enumerate}
    \item\begin{table}[H]
        \centering
        \caption{二元码}
        \begin{tabular}{c|c|c|c|c|c|c|c|c|c|c}
            \hline
            字母 & $a_1$ & $a_2$ & $a_3$ & $a_4$ & $a_5$ & $a_6$ & $a_7$ & $a_8$ & $a_9$ & $a_{10}$ \\
            \hline
            概率 $P$ & 0.16 & 0.14 & 0.13 & 0.12 & 0.1 & 0.09 & 0.08 & 0.07 & 0.06 & 0.05 \\
            编码长度 $l$ & 3 & 3 & 3 & 3 & 3 & 3 & 4 & 4 & 4 & 4 \\
            编码 & 111 & 101 & 100 & 011 & 001 & 000 & 1101 & 1100 & 0101 & 0100 \\
            \hline
        \end{tabular}
    \end{table}

    平均编码长度为
    \begin{align}
        average(l) &= \sum_{i=1}^{10} P(a_i)l(a_i) \\
        &= 3.26
    \end{align}

    编码效率为
    \begin{align}
        \eta &= \frac{H(A)}{average(l)} \\
        &= \frac{-\sum_{i=1}^{10} P(a_i)\log_2 P(a_i)}{average(l)} \\
        &\approx \frac{3.2344}{3.26} \\
        &\approx 0.9921
    \end{align}

    \item\begin{table}[H]
        \centering
        \caption{三元码}
        \begin{tabular}{c|c|c|c|c|c|c|c|c|c|c}
            \hline
            字母 & $a_1$ & $a_2$ & $a_3$ & $a_4$ & $a_5$ & $a_6$ & $a_7$ & $a_8$ & $a_9$ & $a_{10}$ \\
            \hline
            概率 $P$ & 0.16 & 0.14 & 0.13 & 0.12 & 0.1 & 0.09 & 0.08 & 0.07 & 0.06 & 0.05 \\
            编码长度 $l$ & 2 & 2 & 2 & 2 & 2 & 2 & 2 & 2 & 3 & 3 \\
            编码 & 22 & 21 & 20 & 12 & 10 & 02 & 01 & 00 & 111 & 110 \\
            \hline
        \end{tabular}
    \end{table}

    第一次合并节点数为
    \begin{align}
        & ((10 - 2) \bmod (3 - 1)) + 2 \\
        =& 2
    \end{align}

    平均编码长度为
    \begin{align}
        average(l) &= \sum_{i=1}^{10} P(a_i)l(a_i) \\
        &= 2.11
    \end{align}

    编码效率为
    \begin{align}
        \eta &= \frac{H_3(A)}{average(l)} \\
        &= \frac{-\sum_{i=1}^{10} P(a_i)\log_3 P(a_i)}{average(l)} \\
        &\approx \frac{2.0407}{2.11} \\
        &\approx 0.9671
    \end{align}
\end{enumerate}

\paragraph{题目 2}

\begin{enumerate}
    \item\begin{table}[H]
        \centering
        \caption{二元码}
        \begin{tabular}{c|c|c|c}
            \hline
            字母 & $a_1$ & $a_2$ & $a_3$ \\
            \hline
            概率 $P$ & 0.5 & 0.3 & 0.2 \\
            编码长度 $l$ & 1 & 2 & 2 \\
            编码 & 0 & 11 & 10 \\
            \hline
        \end{tabular}
    \end{table}

    平均编码长度为
    \begin{align}
        average(l) &= \sum_{i=1}^3 P(a_i)l(a_i) \\
        &= 1.5
    \end{align}

    编码效率为
    \begin{align}
        \eta &= \frac{H(A)}{average(l)} \\
        &= \frac{-\sum_{i=1}^3 P(a_i)\log_2 P(a_i)}{average(l)} \\
        &\approx \frac{1.4855}{1.5} \\
        &\approx 0.9903
    \end{align}

    \item\begin{table}[H]
        \centering
        \caption{二元码}
        \begin{tabular}{c|c|c|c|c|c|c|c|c|c}
            \hline
            字母 & $a_1a_1$ & $a_1a_2$ & $a_1a_3$ & $a_2a_1$ & $a_2a_2$ & $a_2a_3$ & $a_3a_1$ & $a_3a_2$ & $a_3a_3$ \\
            \hline
            概率 $P$ & 0.25 & 0.15 & 0.1 & 0.15 & 0.09 & 0.06 & 0.1 & 0.06 & 0.04 \\
            编码长度 $l$ & 2 & 3 & 3 & 3 & 4 & 4 & 3 & 4 & 4 \\
            编码 & 10 & 110 & 001 & 111 & 0111 & 0101 & 000 & 0110 & 0100 \\
            \hline
        \end{tabular}
    \end{table}

    平均编码长度为
    \begin{align}
        average(l) &= \sum_{i=1}^3\sum_{j=1}^3 P(a_ia_j)l(a_ia_j) \\
        &= 3.0
    \end{align}

    编码效率为
    \begin{align}
        \eta &= \frac{H(A)}{average(l)} \\
        &= \frac{-\sum_{i=1}^3\sum_{j=1}^3 P(a_ia_j)\log_2 P(a_ia_j)}{average(l)} \\
        &\approx \frac{2.9710}{3.0} \\
        &\approx 0.9903
    \end{align}

    \item\begin{table}[H]
        \centering
        \caption{二元码}
        \begin{tabular}{c|c|c|c|c|c|c|c|c|c}
            \hline
            字母 & $a_1a_1a_1$ & $a_1a_1a_2$ & $a_1a_1a_3$ & $a_1a_2a_1$ & $a_1a_2a_2$ & $a_1a_2a_3$ & $a_1a_3a_1$ & $a_1a_3a_2$ & $a_1a_3a_3$ \\
            \hline
            概率 $P$ & 0.125 & 0.075 & 0.05 & 0.075 & 0.045 & 0.03 & 0.05 & 0.03 & 0.02 \\
            编码长度 $l$ & 3 & 4 & 4 & 4 & 4 & 5 & 4 & 5 & 6 \\
            编码 & 100 & 1100 & 0011 & 1101 & 0000 & 01101 & 0010 & 01111 & 111100 \\
            \hline
            字母 & $a_2a_1a_1$ & $a_2a_1a_2$ & $a_2a_1a_3$ & $a_2a_2a_1$ & $a_2a_2a_2$ & $a_2a_2a_3$ & $a_2a_3a_1$ & $a_2a_3a_2$ & $a_2a_3a_3$ \\
            \hline
            概率 $P$ & 0.075 & 0.045 & 0.03 & 0.045 & 0.027 & 0.018 & 0.03 & 0.018 & 0.012 \\
            编码长度 $l$ & 4 & 4 & 5 & 5 & 5 & 6 & 5 & 6 & 7 \\
            编码 & 1011 & 0001 & 01110  & 11111 & 01010 & 101010 & 10100 & 111000 & 1111011 \\
            \hline
            字母 & $a_3a_1a_1$ & $a_3a_1a_2$ & $a_3a_1a_3$ & $a_3a_2a_1$ & $a_3a_2a_2$ & $a_3a_2a_3$ & $a_3a_3a_1$ & $a_3a_3a_2$ & $a_3a_3a_3$ \\
            \hline
            概率 $P$ & 0.05 & 0.03 & 0.02 & 0.03 & 0.018 & 0.012 & 0.02 & 0.012 & 0.008 \\
            编码长度 $l$ & 4 & 5 & 6 & 5 & 6 & 7 & 6 & 7 & 7 \\
            编码 & 0100 & 01100 & 111011 & 01011 & 101011 & 1111010 & 111001 & 1110101 & 1110100 \\
            \hline
        \end{tabular}
    \end{table}

    平均编码长度为
    \begin{align}
        average(l) &= \sum_{i=1}^3\sum_{j=1}^3\sum_{k=1}^3 P(a_ia_ja_k)l(a_ia_ja_k) \\
        &= 6.0
    \end{align}

    编码效率为
    \begin{align}
        \eta &= \frac{H(A)}{average(l)} \\
        &= \frac{-\sum_{i=1}^3\sum_{j=1}^3\sum_{k=1}^3 P(a_ia_ja_k)\log_2 P(a_ia_ja_k)}{average(l)} \\
        &\approx \frac{5.9419}{6.0} \\
        &\approx 0.9903
    \end{align}
\end{enumerate}

\subparagraph{实际问题}

第二题中随着字母数量的增多,平均编码长度与信息熵应该同步增加,并使得编码效率不变。在第二题的第一小问与第二小文中,这显然成立。可是在第三小问的实际计算中,发现

\begin{align}
    average(l) &= \sum_{i=1}^3\sum_{j=1}^3\sum_{k=1}^3 P(a_ia_ja_k)l(a_ia_ja_k) \\
    &\approx 4.486999999999999 \\
    \\
    H(A) &= -\sum_{i=1}^3\sum_{j=1}^3\sum_{k=1}^3 P(a_ia_ja_k)\log_2(a_ia_ja_k) \\
    &\approx 4.456425891682005 \\
    \\
    \eta &= \frac{H(A)}{average(l)} \\
    &\approx 0.9931860690176078
\end{align}

反复检查多次,不知道问题出在了那里,请老师答疑解惑。

\end{document}
