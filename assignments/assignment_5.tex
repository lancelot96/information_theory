\documentclass{ctexart}

\usepackage{color}
\usepackage{float}
\usepackage{amsmath}
\usepackage{geometry}
\usepackage[colorlinks]{hyperref}

\geometry{top=2.5cm, bottom=2.5cm, left=2.5cm, right=2.5cm}

\title{信息论作业 5}
\author{史泽宇}
\date{\today}

\begin{document}

\maketitle

\paragraph{题目 1}

\begin{enumerate}
    \item 为了满足题目排序条件,对字母进行重排
    \begin{table}[H]
        \centering
        \caption{二元码}
        \begin{tabular}{ccccccccc}
            \hline
            字母 & $a_8$ & $a_7$ & $a_6$ & $a_5$ & $a_4$ & $a_3$ & $a_2$ & $a_1$ \\
            \hline
            概率 & $\frac{1}{16}$ & $\frac{1}{16}$ & $\frac{1}{16}$ & $\frac{1}{16}$ & $\frac{1}{8}$ & $\frac{1}{8}$ & $\frac{1}{4}$ & $\frac{1}{4}$ \\
            $cdf$ & 0 & $\frac{1}{16}$ & $\frac{2}{16}$ & $\frac{3}{16}$ & $\frac{4}{16}$ & $\frac{6}{16}$ & $\frac{8}{16}$ & $\frac{12}{16}$ \\
            $I$ & 4 & 4 & 4 & 4 & 3 & 3 & 2 & 2 \\
            $n$ & 4 & 4 & 4 & 4 & 3 & 3 & 2 & 2 \\
            编码 & 0000 & 0001 & 0010 & 0011 & 010 & 011 & 10 & 11 \\
            \hline
        \end{tabular}
    \end{table}
    \item\begin{enumerate}
        \item 首先证明该编码是异字头码。$\forall a_k$,有 $n_k = \lceil \log_2\frac{1}{P(a_k)} \rceil \geq \log_2\frac{1}{P(a_k)}$,所以有 $2^{-n_k} \leq P(a_k) = cdf(a_{k+1}) - cdf(a_k) \leq cdf(a_{k+n}) - cdf(a_k), where\ n \geq 1$。即 $a_{k+n}$ 与 $a_k$ 的编码至少有一位不同
        \item $\bar{n} = \sum_{a \in U} P(a)\lceil\log_2\frac{1}{P(a)}\rceil \geq \sum_{a \in U} P(a)\log_2\frac{1}{P(a)} = H(U) bits$ \\
        $\bar{n} = \sum_{a \in U} P(a)\lceil\log_2\frac{1}{P(a)}\rceil < \sum_{a \in U} P(a)\log_2\frac{1}{P(a)} + 1 = (H(U) + 1) bits$
        % 不等号可以取到吗
    \end{enumerate}
\end{enumerate}

\paragraph{题目 2}

\begin{enumerate}
    \item 由于字母出现的概率相等,所以当 $\alpha = 1$ 时,对应的编码树为满二叉树,编码长度为 $j$。当 $\alpha = 2$ 时,编码长度为 $j + 1$。当 $1 < \alpha < 2$ 时,由于取整编码长度向上取证,所以只存在 $j, j + 1$ 长度的编码
    \item 设长为 $j$ 的码字个数为 $N_j$,长度为 $j + 1$ 的码字数目为 $N_{j+1}$,根据二元 Huffman编码思想(必定占满整个码树),即 $N_j + N_{j+1} = K = \alpha2^j, N_j2^j + N_{j+1}2^{-(j+1)} = 1$。从而 $N_j = (2 - \alpha)2^j, N_{j+1} = (\alpha - 1)2^{j+1}$
    \item $l = \frac{jN_j + (j + 1)N_{j+1}}{K} = \frac{(2  -\alpha)j + 2(\alpha - 1)(j + 1)}{\alpha} = \frac{2j - j\alpha + 2j\alpha + 2\alpha - 2j - 2}{\alpha} = \frac{j\alpha + 2\alpha - 2}{\alpha} = j + 2 - \frac{2}{\alpha}$
\end{enumerate}

\paragraph{题目 3}

\begin{enumerate}
    \item\begin{table}[H]
        \centering
        \caption{算术编码}
        \begin{tabular}{cccc}
            \hline
            步骤 & 输入字母 & 编码区间 & 间隔 \\
            \hline
            1 & 1 & [0.25, 1) & 0.75 \\
            2 & 0 & [0.25, 0.4375) & 0.1875 \\
            3 & 1 & [0.296875, 0.4375) & 0.140625 \\
            4 & 1 & [0.300390625, 0.4375) & 0.10546875 \\
            5 & 0 & [0.33203125, 0.3583984375) & 0.0263671875 \\
            6 & 1 & [0.338623046875, 0.3583984375) & 0.019775390625 \\
            7 & 1 & [0.34356689453125, 0.3583984375) & 0.01483154296875 \\
            8 & 1 & [0.3472747802734375, 0.3583984375) & 0.0111236572265625 \\
            9 & 1 & [0.3500556945800781, 0.3583984375) & 0.008342742919921875 \\
            10 & 0 & [0.3500556945800781, 0.3521413803100586) & 0.0020856857299804688 \\
            11 & 1 & [0.35057711601257324, 0.3521413803100586) & 0.0015642642974853516 \\
            12 & 1 & [0.3509681820869446, 0.3521413803100586) & 0.0011731982231140137 \\
            13 & 0 & [0.3509681820869446, 0.3512614816427231) & 0.0002932995557785034 \\
            14 & 1 & [0.3510415069758892, 0.3512614816427231) & 0.00021997466683387756 \\
            15 & 1 & [0.3510965006425977, 0.3512614816427231) & 0.00016498100012540817 \\
            16 & 1 & [0.351137745892629, 0.3512614816427231) & 0.00012373575009405613 \\
            \hline
        \end{tabular}
    \end{table}
    编码长度为 $n = \lceil -\log_2 0.00012373575009405613 \rceil = 13$,而 $0.351137745892629 \approx (0.0101100111100100)_2$,所以编码结果为 0101100111101。编码效率为 $\frac{H(U)}{\frac{13}{16}} = \frac{\frac{1}{4}\log_2 4 + \frac{3}{4}\log_2\frac{4}{3}}{\frac{13}{16}} \approx 0.9985$
    \item\begin{table}[H]
        \centering
        \caption{LZ 编码}
        \begin{tabular}{cccc}
            \hline
            段号 & 前一段号 & 最后一位编码 & 段编码 \\
            \hline
            1 & 0 & 1 & 0001 \\
            2 & 0 & 0 & 0000 \\
            3 & 1 & 1 & 0011 \\
            4 & 2 & 1 & 0101 \\
            5 & 3 & 1 & 0111 \\
            6 & 4 & 1 & 1001 \\
            7 & 6 & 1 & 1101 \\
            \hline
        \end{tabular}
    \end{table}
    所以最终编码为 0001000000110101011110011101,编码效率为 $\frac{H(U)}{\frac{28}{16}} \approx 0.4636$
\end{enumerate}

\paragraph{题目 4}\textcolor{red}{第二版的作业中弃用}

题目似乎有点问题,如果按照题目计算的话,结果为

设信源输出长度为 $l$,则编码长度为 $\bar{l} = 1\frac{l}{2} + 2\frac{l}{4} + 3\frac{l}{8} + 4\frac{l}{8} = \frac{15}{8}l$,0 的个数为 $\bar{l}_0 = \frac{l}{2} + \frac{l}{4} + \frac{l}{8} + \frac{l}{8} = l, \bar{l}_1 = \frac{l}{4} + 2\frac{l}{8} + 3\frac{l}{8} = \frac{7}{8}l, P(0) = \frac{8}{15}, P(1) = \frac{7}{15}$

教材上课后题目中,最后一个字母的编码为 111,使用教材上的编码计算结果为

设信源输出长度为 $l$,则编码长度为 $\bar{l} = 1\frac{l}{2} + 2\frac{l}{4} + 3\frac{l}{8} + 3\frac{l}{8} = \frac{7}{4}l$,0 的个数为 $\bar{l}_0 = \frac{l}{2} + \frac{l}{4} + \frac{l}{8} = \frac{7}{8}l, \bar{l}_1 = \frac{l}{4} + 2\frac{l}{8} + 3\frac{l}{8} = \frac{7}{8}l, P(0) = \frac{1}{2}, P(1) = \frac{1}{2}$

\paragraph{题目 5}

\begin{table}[H]
    \centering
    \caption{信源概率分布}
    \label{tab.5}
    \begin{tabular}{ccc}
        \hline
        字母 & 概率 & 字母长度 \\
        \hline
        1 & 0.1 & 1 \\
        01 & (0.9)(0.1) & 2 \\
        001 & $(0.9)^2(0.1)$ & 3 \\
        0001 & $(0.9)^3(0.1)$ & 4 \\
        00001 & $(0.9)^4(0.1)$ & 5 \\
        000001 & $(0.9)^5(0.1)$ & 6 \\
        0000001 & $(0.9)^6(0.1)$ & 7 \\
        00000001 & $(0.9)^7(0.1)$ & 8 \\
        00000000 & $0.9^8$ & 8 \\
        \hline
    \end{tabular}
\end{table}

\begin{enumerate}
    \item 由信源的概率分布可计算 $H(U) \approx 0.469 bits$
    \item 由表 \ref{tab.5} 中的概率分布与字母长度可计算 $\bar{n}_1 \approx 5.6953$
    \item 由表 \ref{tab.5} 中的概率分布可计算 $\bar{n}_2 \approx 2.7086$
    \item 观察编码结果发现,任意编码都不是其他编码的前缀,所以是异字头码
\end{enumerate}

\paragraph{题目 6}

\begin{enumerate}
    \item 显然是三元对称信道,输入等概时达到信道容量 $C = \log_2 3 - H(P) bps$
    \item 显然是输入为二元输出为四元的对称信道,输入等概时达到信道容量 $C = \log_2 4 + (1 - p)\log_2\frac{(1 - p)}{2} + p\log_2\frac{p}{2} = 1 - H(P) bps$
    \item 可以拆分成两个和信道的组合,$C_1 = 1 - H(P), C_2 = 0, C = \log_2(2^{C_1} + 2^{C_2}) = \log_2(2^{1 - H(P)} + 1) bps$
\end{enumerate}

\paragraph{题目 7}

\begin{enumerate}
    \item\begin{equation}
        \begin{bmatrix}
            \frac{3}{4} & \frac{1}{4} & 0 \\
            \frac{1}{3} & \frac{1}{3} & \frac{1}{3} \\
            0 & \frac{1}{4} & \frac{3}{4} \\
        \end{bmatrix}
    \end{equation}
    可以简化为如下的二元纯删除信道
    \begin{equation}
        \begin{bmatrix}
            \frac{3}{4} & \frac{1}{4} & 0 \\
            0 & \frac{1}{4} & \frac{3}{4} \\
        \end{bmatrix}
    \end{equation}
    显然达到信道容量的分布为等概分布,信道容量为 $C = \frac{3}{4} bps$。
    \item\begin{equation}
        \begin{bmatrix}
            \frac{1}{3} & \frac{1}{3} & 0 & \frac{1}{3} \\
            0 & \frac{1}{3} & \frac{1}{3} & \frac{1}{3} \\
            \frac{1}{3} & 0 & \frac{1}{3} & \frac{1}{3} \\
        \end{bmatrix}
    \end{equation}
    显然为准对称信道,输入等概时达到信道容量,可以拆分为以下两个弱对称信道\footnote{\href{http://shannon.cm.nctu.edu.tw/it/c1-4s13.pdf}{Lemma 4.18}}
    \begin{equation}
        \begin{bmatrix}
            \frac{1}{3} & \frac{1}{3} & 0 \\
            0 & \frac{1}{3} & \frac{1}{3} \\
            \frac{1}{3} & 0 & \frac{1}{3} \\
        \end{bmatrix}
        \begin{bmatrix}
            \frac{1}{3} \\
            \frac{1}{3} \\
            \frac{1}{3} \\
        \end{bmatrix}
    \end{equation}
    $C_1 = \log_2 3 - H(\frac{\frac{1}{3}}{\frac{2}{3}}, \frac{\frac{1}{3}}{\frac{2}{3}}, \frac{0}{\frac{2}{3}}) = \log_2\frac{3}{2}, C_2 = \log_2 1 - H(\frac{\frac{1}{3}}{\frac{1}{3}}) = 0, C = \frac{2}{3}C_1 + \frac{1}{3}C_2 = \frac{2}{3}\log_2\frac{3}{2} bps$
\end{enumerate}

\paragraph{题目 8}

信道容量为 $C = W\log_2(1 + \frac{S}{N}) = 3000 \log_2(1 + 1000) \approx 30 kbps$,三十分钟传送的信息为 $3 * 60 * C \approx 5.3823 mb$

\paragraph{题目 9}

\begin{enumerate}
    \item 最大似然译码
    \begin{equation}
        \begin{bmatrix}
            P(y_i|x_j) & y_1 & y_2 & y_3 \\
            x_1 & \textcolor{red}{\frac{1}{2}} & \frac{1}{3} & \frac{1}{6} \\
            x_2 & \frac{1}{6} & \textcolor{red}{\frac{1}{2}} & \frac{1}{3} \\
            x_3 & \frac{1}{3} & \frac{1}{6} & \textcolor{red}{\frac{1}{2}} \\
        \end{bmatrix}
    \end{equation}
    由矩阵中的最大转移概率可得译码规则为
    \begin{equation}
        \begin{split}
            F(y_1) &= x_1 \\
            F(y_2) &= x_2 \\
            F(y_3) &= x_3
        \end{split}
    \end{equation}
    误码率为 $P_e = (\frac{1}{2} + \frac{1}{4} + \frac{1}{4})(\frac{1}{6} + \frac{1}{3}) = \frac{1}{2}$
    \item 最大后验译码,由贝叶斯公式可得
    \begin{equation}
        \begin{bmatrix}
            P(x_i|y_j) & y_1 & y_2 & y_3 \\
            x_1 & \textcolor{red}{\frac{2}{3}} & \textcolor{red}{\frac{1}{2}} & \frac{2}{7} \\
            x_2 & \frac{1}{9} & \frac{3}{8} & \frac{2}{7} \\
            x_3 & \frac{2}{9} & \frac{1}{8} & \textcolor{red}{\frac{3}{7}} \\
        \end{bmatrix}
    \end{equation}
    由矩阵中的最大后验概率可得译码规则为
    \begin{equation}
        \begin{split}
            F(y_1) &= x_1 \\
            F(y_2) &= x_1 \\
            F(y_3) &= x_3
        \end{split}
    \end{equation}
    误码率为 $P_e = (\frac{1}{2})(\frac{1}{6})+ \frac{1}{4} + \frac{1}{4}(\frac{1}{3} + \frac{1}{6}) = \frac{11}{24}$
\end{enumerate}

\end{document}
