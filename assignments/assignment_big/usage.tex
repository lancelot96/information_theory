\documentclass{ctexart}

\usepackage{geometry}
\geometry{top=2.5cm, bottom=2.5cm, left=2.5cm, right=2.5cm}

\title{信息论及其应用实验一:数据压缩 \\ 使用手册}
\author{史泽宇}
\date{\today}

\begin{document}

\maketitle

\section{概述}

本实现使用 Rust 程序设计语言完成了 LZ 编码,Huffman 编码以及不完美的算术编码,总共代码行数为 1012 行。实现包括了一个简单易用的命令行接口,可以在命令行下调用本程序,以下是使用说明。

\section{使用手册}

\subsection{主界面}

直接使用命令 assignment\_big,或者提供 -h, --help 参数可以获取程序的使用说明

\begin{verbatim}
    assignment_big 0.1.0
    yuyy <lancelot96@protonmail.ch>

    USAGE:
        assignment_big <SUBCOMMAND>

    FLAGS:
        -h, --help       Prints help information
        -V, --version    Prints version information

    SUBCOMMANDS:
        arithmetic    
        help          Prints this message or the help of the given subcommand(s)
        huffman       
        lz
\end{verbatim}

主要提供了三个子命令,lz,huffman,arithmetic,分别表示使用哪种编码算法。每个子命令后都带有几个选项,其中共有的几个选项是 --source <source>,--target <target>,<--encode|--decode>。source 表示输入文件路径,target 表示输出文件路径,encode 表示对输入文件进行编码,decode 表示对输入文件进行解码。

\subsection{lz 子命令}

使用 assignment\_big lz 调用 lz 子命令,表示使用 LZ 编码算法

\begin{verbatim}
    error: The following required arguments were not provided:
        --alphabet <alphabet>
        --letter-code-length <letter-code-length>
        --prefix-code-length <prefix-code-length>
        --source <source>
        --target <target>
        <--encode|--decode>

    USAGE:
        assignment_big lz --alphabet <alphabet> --base <base> --letter-code-length
        <letter-code-length> --prefix-code-length <prefix-code-length> --source <source>
        --target <target> <--encode|--decode>

    For more information try --help
\end{verbatim}

对于编码模式,只需要另外提供字母表参数即可,如

\begin{verbatim}
    assignment_big lz --encode --source data.txt --target lz_coded \
    --alphabet alphabet.txt
\end{verbatim}

对于解码模式,除了需要提供字母表参数,还需要提供编码长度

\begin{verbatim}
    assignment_big lz --decode --source lz_coded.txt --target lz_decoded.txt \
    --alphabet alphabet.txt  --letter-code-length 7 --prefix-code-length 11
\end{verbatim}

\subsection{huffman 子命令}

使用 assignment\_big huffman 调用 huffman 子命令,表示使用 Huffman 编码算法

\begin{verbatim}
    error: The following required arguments were not provided:
        --codebook <codebook>
        --source <source>
        --target <target>
        <--encode|--decode>

    USAGE:
        assignment_big huffman --base <base> --codebook <codebook> --source <source>
        --target <target> <--encode|--decode>

    For more information try --help
\end{verbatim}

对于编码模式,不需要提供任何额外参数

\begin{verbatim}
    assignment_big huffman --encode --source ./data.txt --target huffman_coded
\end{verbatim}

对于解码模式,还需要提供编码之后产生的编码映射

\begin{verbatim}
    assignment_big huffman --decode --source ./huffman_coded.txt --target \
    huffman_decoded.txt --codebook huffman_coded.cb
\end{verbatim}

\subsection{arithmetic 子命令}

使用 assignment\_big arithmetic 调用 arithmetic 子命令,表示使用算术编码算法

\begin{verbatim}
    error: The following required arguments were not provided:
        --distribution <distribution>
        --source <source>
        --target <target>
        <--encode|--decode>

    USAGE:
        assignment_big arithmetic --base <base> --distribution <distribution>
        --source <source> --target <target> <--encode|--decode>

    For more information try --help
\end{verbatim}

对于编码模式,不需要提供任何额外参数

\begin{verbatim}
    assignment_big arithmetic --encode --source data.txt --target arithmetic_coded
\end{verbatim}

对于解码模式,还需要提供信源的概率分布

\begin{verbatim}
    assignment_big arithmetic --decode --source ./arithmetic_coded.txt \
    --target arithmetic_coded_decoded.txt --distribution distribution.prob
\end{verbatim}

但是需要注意,本程序实现的算术编码不能处理较长内容的输入,具体可以参考测试文件中的内容。

\end{document}
