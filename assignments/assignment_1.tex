\documentclass{ctexart}

\usepackage{amsmath}
\usepackage{geometry}

\newtheorem{cl}{Corollary}
\geometry{left=2.5cm,right=2.5cm,top=2.5cm,bottom=2.5cm}

\title{信息论作业 1}
\author{史泽宇}
\date{\today}

\begin{document}

\maketitle

\paragraph{题目 1}
\begin{enumerate}
    \item 两枚骰子总点数之和为 7 的情况有 $\{(1, 6), (2, 5), (3, 4), (4, 3), (5, 2), (6, 1)\}$ 6 种情况,总共可能产生 $6 \times 6 = 36$ 种情况。所以投掷总点数之和为 7 的概率为 $\frac{3}{18}$,其自信息为 $\log_2\frac{18}{3} \approx 2.58 bits$。
    \item 两枚骰子总点数之和为 12 的情况有 $\{(1, 1)\}$ 1 种情况。所以投掷总点数之和为 12 的概率为 $\frac{1}{36}$,其自信息为 $\log_2 36 \approx 5.17 bits$。
\end{enumerate}

\paragraph{题目 2}
本题目如果直接计算多个离散概率的互信息,脑补了一下计算复杂度,可能会比较高,查阅了部分资料后\cite{madiman2008entropy},得到如下结论(论文上说 Elements of Information Theory 上面有,但是我没找到):

\begin{cl}
Let $Y = X_1 + X_2$, where $X_1$ and $X_2$ are independent, then $I(X_1; Y) = H(Y) - H(X_2)$
\end{cl}

\begin{align}
    I(X_1; X_1 + X_2) &= H(X_1 + X_2) - H(X_1 + X_2|X_1) \\
    &= H(X_1 + X_2) - H(X_2|X_1) \\
    &= H(X_1 + X_2) - H(X_2) \\
    &= H(Y) - H(X_2)
\end{align}

其中 (1) 到 (2) 的步骤不是很明白,想请老师答疑解惑一下。本题目的解答中很大程度上依靠了 (1) 到 (2) 化简方法。这里令 $X_1$ 表示第一颗骰子的结果,$X_2$ 表示第二颗骰子的结果,$X_3$ 表示第三颗骰子的结果,所以 $X = X_1, Y = X_1 + X_2, Z = X_1 + X_2 + X_3$。

\begin{align}
    H(X) &= H(X_1) = H(X_2) = H(X_3) \\
    &= \sum_1^6 \frac{1}{6}\log_2 6 \\
    &= \log_2 6 bits \\
    &\approx 2.5850 bits
\end{align}

\begin{align}
    H(Y) &= H(X_1 + X_2) = H(X_2 + X_3) = H(X_1 + X_3) \\
    &= \sum_p^\Omega p\log_2 \frac{1}{p} \\
    &\approx 3.2744 bits \\
    \Omega &= \{\frac{1}{36}, \frac{2}{36}, \frac{3}{36}, \frac{4}{36}, \frac{5}{36}, \frac{6}{36}, \frac{5}{36}, \frac{4}{36}, \frac{3}{36}, \frac{2}{36}, \frac{1}{36}\} 
\end{align}

\begin{enumerate}
    \item\begin{align}
        H(Z|Y) &= H(X_1 + X_2 + X_3|X_1 + X_2) \\
        &= H(X_3|X_1 + X_2) \\
        &= H(X_3) \\
        &\approx 2.5850 bits
    \end{align}
    \item\begin{align}
        H(X|Y) &= H(X_1|X_1 + X_2) \\
        &= H(X_1) - I(X_1; X_1 + X_2) \\
        &= H(X_1) - H(X_1 + X_2) + H(X_2) \\
        &\approx 1.8955 bits
    \end{align}
    \item\begin{align}
        H(Z|X, Y) &= H(X_1 + X_2 + X_3|X_1, X_1 + X_2) \\
        &= H(X_3|X_1, X_1 + X_2) \\
        &= H(X_3) \\
        &\approx 2.5850 bits
    \end{align}
    \item\begin{align}
        H(X, Z|Y) &= H(X|Y) + H(Z|X, Y) \\
        &\approx 4.4805 bits
    \end{align}
    \item\begin{align}
        H(Z|X) &= H(X_1 + X_2 + X_3|X_1) \\
        &= H(X_2 + X_3|X_1) \\
        &= H(X_2 + X_3) \\
        &\approx 3.2744 bits
    \end{align}
\end{enumerate}

\paragraph{题目 3} 

\begin{enumerate}
    \item\begin{align}
        P(0) &= \sum_{i=1}^8 P(u_i)P(0|u_i) \\
        &= \frac{4(1 - p)}{8} + \frac{4p}{8} \\
        &= \frac{1}{2} \\
        P(u_1|0) &= \frac{P(u_1)P(0|u_1)}{P(0)} \\
        &= \frac{\frac{1}{8}(1 - p)}{\frac{1}{2}} \\
        &= \frac{1 - p}{4} \\
        I(u_1; 0) &= \log_2 \frac{P(u_1|0)}{P(u_1)} \\
        &= \log_2 \frac{\frac{1 - p}{4}}{\frac{1}{8}} \\
        &= 1 + \log_2 (1 - p) bits
    \end{align}
    \item\begin{align}
        P(00) &= \sum_{i=1}^8 P(u_i)P(00|u_i) \\
        &= \frac{1}{4} \\
        P(u_1|00) &= \frac{P(u_1)P(00|u_1)}{P(00)} \\
        &= \frac{{1 - p}^2}{2} \\
        I(u_1; 00) &= \log_2 \frac{P(u_1|00)}{P(u_1)} \\
        &= 2 + 2\log_2 (1 - p) bits
    \end{align}
    \item\begin{align}
        P(000) &= \sum_{i=1}^8 P(u_i)P(000|u_i) \\
        &= \frac{1}{8} \\
        P(u_1|000) &= \frac{P(u_1)P(000|u_1)}{P(000)} \\
        &= (1 - p)^3 \\
        I(u_1; 000) &= \log_2 \frac{P(u_1|000)}{P(u_1)} \\
        &= 3 + 3\log_2 (1 - p) bits
    \end{align}
    \item\begin{align}
        P(0000) &= \sum_{i=1}^8 P(u_i)P(0000|u_i) \\
        &= \frac{(1 - p)^4 + p^4 + 6(1 - p)^2p^2}{8} \\
        P(u_1|0000) &= \frac{P(u_1)P(0000|u_1)}{P(0000)} \\
        &= \frac{(1 - p)^4}{(1 - p)^4 + p^4 + 6(1 - p)^2p^2} \\
        I(u_1; 0000) &= \log_2 \frac{P(u_1|0000)}{P(u_1)} \\
        &= \log_2 \frac{8(1 - p)^4}{(1 - p)^4 + p^4 + 6(1 - p)^2p^2} bits
    \end{align}
\end{enumerate}

\bibliographystyle{plain}
\bibliography{assignment_1} 

\end{document}
