\documentclass{ctexart}

\usepackage{amsmath}
\usepackage{geometry}

\geometry{top=2.5cm, bottom=2.5cm, left=2.5cm, right=2.5cm}

\title{信息论作业 3}
\author{史泽宇}
\date{\today}

\begin{document}

\maketitle

\paragraph{题目 1}

\begin{enumerate}
    \item\begin{align}
        P(Y|x=1) &= \begin{cases}
            \frac{1}{4} & -1 < y \le 3 \\
            0 &
        \end{cases} \\
        P(Y|x=-1) &= \begin{cases}
            \frac{1}{4} & -3 < y \le 1 \\
            0 &
        \end{cases} \\
        \omega(Y) &= \sum_x P(x)P(Y|x) \\
        &= \begin{cases}
            \frac{1}{8} & -3 < y \le -1 \\
            \frac{1}{4} & -1 < y \le 1 \\
            \frac{1}{8} & 1 < y \le 3 \\
            0 &
        \end{cases}
    \end{align}
    \item\begin{align}
        I(X; Y) &= H_c(Y) - H_c(Y|X) \\
        &= -\int_y P(y)\log_2 P(y)dy + \int_{-2}^2 P(x)P(y|x)\log_2 P(y|x)dy \\
        &= -2\int_1^3\frac{1}{8}\log_2\frac{1}{8}dy - \int_{-1}^1\frac{1}{4}\log_2\frac{1}{4} + \int_{-2}^2\sum_x P(x)P(y|x)\log_2 P(y|x)dy \\
        &= \frac{12}{8} + 1 + \int_{-2}^2 \frac{1}{4}\log_2\frac{1}{4}dy \\
        &= 0.5 bit
    \end{align}
    \item\begin{align}
        P(V) &= \begin{cases}
            \frac{1}{4} & y \le -1 \\
            \frac{1}{2} & -1 < y \le 1 \\
            \frac{1}{4} & y > 1
        \end{cases} \\
        P(V|x=1) &= \begin{cases}
            0 & y \le -1 \\
            \frac{1}{2} & -1 < y \le 1 \\
            \frac{1}{2} & y > 1
        \end{cases} \\
        P(V|x=-1) &= \begin{cases}
            \frac{1}{2} & y \le -1 \\
            \frac{1}{2} & -1 < y \le 1 \\
            0 & y > 1
        \end{cases} \\
        I(X; V) &= H(V) - H(V|X) \\
        &= -\sum_v P(v)\log_2 P(v) + \sum_x\sum_v P(x, v)\log_2 P(v|x) \\
        &= 1.5 + \log_2\frac{1}{2} \\
        &= 0.5 bit
    \end{align}
    由于 $P(V)$ 是 $\omega(Y)$ 的积分,所以两者信息量相同。或者由数据处理不等式,$I(X|Y) \ge I(X|V)$,而 $P(V)$ 与 $\omega(Y)$ 互为可逆函数,所以等号成立,两者信息量相同。
\end{enumerate}

\paragraph{题目 2}

\begin{enumerate}
    \item\begin{align}
        & \lceil\log_2 (C_{100}^0 + C_{100}^1 + C_{100}^2)\rceil \\
        =& \lceil\log_2 (1 + 100 + 4950)\rceil \\
        =& 13
    \end{align}
    \item\begin{align}
        P_e &= 1 - C_{100}^0 P(a_2)^{100} - C_{100}^1 P(a_1)P(a_2)^{99} - C_{100}^2 P(a_1)^2P(a_2)^{98} \\
        &\approx 0.00775
    \end{align}
\end{enumerate}

\paragraph{题目 3}

\begin{align}
    H(U) &= -\sum_u P(u)\log_2 P(u) \\
    &= \frac{1}{4}\log_2 4 + \frac{3}{4}\log_2 \frac{4}{3} \\
    &\approx 0.81128 bit
    \\
    \sigma^2 &= \sum_u P(u)(I(u) - H(U))^2 \\
    &= \frac{1}{4}(\log_2 4 - H(U))^2 + \frac{3}{4}(\log_2\frac{4}{3} - H(U))^2 \\
    &\approx 0.47102 bit
\end{align}

\begin{enumerate}
    \item 由切比雪夫不等式可得
    \begin{align}
        L &\ge \frac{\sigma^2}{\varepsilon\delta^2} \\
        &\approx 1884.08 \\
        L_0 &= \lceil L\rceil \\
        &= 1885
    \end{align}
    \item 由切比雪夫不等式可得
    \begin{align}
        L &\ge \frac{\sigma^2}{\varepsilon\delta^2} \\
        &\approx 47101989912979.88 \\
        L_0 &= \lceil L\rceil \\
        &= 47101989912980
    \end{align}
    \item\begin{enumerate}
        \item $\delta = 0.05, \varepsilon = 0.1, L_0 = 1885$
        \begin{align}
            upper\_bound &= \lfloor 2^{L_0(H(U) + \delta)}\rfloor \\
            &= \lfloor 2^{1623.51}\rfloor \\
            lower\_bound &= \lceil (1 - \varepsilon)2^{L_0(H(U) - \delta)}\rceil \\
            &= \lceil 0.9 \times 2^{1435.01}\rceil
        \end{align}
        \item $\delta = 10^{-3}, \varepsilon = 10^{-8}, L_0 = 47101989912980$
        \begin{align}
            upper\_bound &= \lfloor 2^{L_0(H(U) + \delta)}\rfloor \\
            &= \lfloor 2^{38259916024808.39}\rfloor \\
            lower\_bound &= \lceil (1 - \varepsilon)2^{L_0(H(U) - \delta)}\rceil \\
            &= \lceil 0.99999999 \times 2^{38165712044982.43}\rceil
        \end{align}
    \end{enumerate}
\end{enumerate}

话说在看书的时候遇到一个问题,关于切比雪夫不等式的使用。首先列出切比雪夫不等式,对于随机变量 $X$,数学期望 $E(X)$,方差 $DX$:

\begin{equation}
    P[|X - E(X)| \ge \varepsilon] \le \frac{D(X)^2}{\varepsilon^2}
\end{equation}

在书上公式 (3.2.10) 中,$\frac{I(\textbf{u}_L)}{L}$ 表示平均单个字母的自信息,为什么 $\frac{\sigma_I^2}{L\varepsilon^2}$ 比标准的切比雪夫不等式增加了变量 $L$:

\begin{equation}
    P_r[|\frac{I(\textbf{u}_L)}{L} - H(U)| > \varepsilon] < \frac{\sigma_I^2}{L\varepsilon^2} = \delta
\end{equation}

\end{document}
